\section{Introduction}
\label{sec:introduction}
Modern research in deep learning, using convolutional neural networks \cite{Dong15} and variational auto-encoders \cite{Kingma2013, Johnson16}, provide promising applications for better super-resolution imaging (SR).  

The use for SR are vast and important for presenting high-quality content fast in modern digital media with compression schemes due to restricted bandwidth. Online streaming services, e.g.\ Netflix, can provide faster streaming through applying SR on low-resolution movies on the local device. Traditional techniques infer information from patches within the image or between frames to lower the entropy and signal-to-noise ratio. 
% TODO: List more trad, techniques

Following Kingma et al \cite{Kingma2013} we suggest a new SR technique using a variational auto-encoder model, which can infer structure in high-resolution (HR) images from low-resolution (LR) images just from reconstructing the underlying process. We compare the reconstructions against traditional image-processing upscaling.

% TODO Hypothesis: Independent of resolution
