\section{Visualisations}
\label{sec:visualisations}

Using a latent size of $N_{\vec{z}} = 2$, the latent space is visualised for both a downsampling factor of $2$ and $4$ in Figure~\ref{fig:manifold}.

\begin{figure}
    \centering
    % \hspace*{\fill}
    \subbottom[$d = 2$.\label{fig:manifold:downsampling:2}]{
        \includegraphics[width=.45\textwidth]{manifold_bs_ds2_l2_e50}
    }
    \\
    % \hfill
    \subbottom[$d = 4$.\label{fig:manifold:downsampling:4}]{
        \includegraphics[width=.45\textwidth]{manifold_bs_ds4_l2_e50}
    }
    % \hspace*{\fill}
    \caption{The learned data manifold for our model latent size of $N_{\vec{z}} = 2$ for both downsampling factors (see \cite{Kingma2013} for details).}
    \label{fig:manifold}
\end{figure}

% The method was also applied to digits made and binarised by hand as well as made using handwriting typefaces

\section{Implementation}
\label{sec:implementation}

We have implemented our method\footnote{Available at \url{https://github.com/chgroenbech/aml-project}.} in \Python using the modules \Theano, \Lasagne, and \Parmesan.
