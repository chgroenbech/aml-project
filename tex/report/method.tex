\section{Method}
\label{sec:method}

We consider the MNIST dataset \cite{MNIST}, $\vec{X} = \curlies{\vec{X}\order{i}}^N_{i=1}$, of i.i.d. images of handwritten digits. \dots

For the reconstruction of high-resolution (HR) digits from low-resolution (LR) digits in the MNIST dataset, our model assume the same underlying data generating process behind the production of digits to infer the same structure between them. Each variable in $\vec{X}$ are assumed to be drawn from a normal distribution.\dots 
%TODO: Skriv mere beskrivelse af modellen og indsæt model diagram

Using mean-field variational Bayes \cite{Kingma2013}, we build a generative model with an approximated decoding posterior distribution, $\enc{\vec{z}|\vec{X}}$ mapping the observed variables $\vec{X}$ into the unobserved latent variables $\vec{z}$. The generative part of the model is then using the conditional likelihood $\dec{\vec{X},\vec{z}}$ to redo the probabilistic process of producing a digit pixel for pixel. \\  
The model is trained based on a per-pixel loss-function derived from the marginal log-likelihood of each datapoint, $i$:
\begin{equation}
	\log \dec{\vec{X}} = \log \dec{\vec{X}\order{1}, ..., \vec{X}\order{N}} = \sum^N_{i=1} \log \dec{\vec{X}\order{i}} 
\end{equation}
From now on we denote $\vec{X}\order{i} = \vec{x}$ for shorter and uncluttered notation.
The marginal log-likelihood can be decomposed like for the EM-algorithm and variational inference \cite[\S10.2]{Bishop2006}:
\begin{equation}
	\log \dec{\vec{X}\order{i}} = D_{KL}\left( \enc{\vec{z}|\vec{x}}||\dec{\vec{z}|\vec{x}}\right) + \mathcal{L}\left(\vec{\theta},\vec{\phi};\vec{x}\right)
\end{equation} 
Here the first term is the Kullback-Leibler divergence, which is a non-negative entropy measure of how much the 2 distributions differ.
\begin{equation}
	D_{KL}\left( \enc{\vec{z}|\vec{x}}||\dec{\vec{z}|\vec{x}}\right) = \int \enc{\vec{z}|\vec{x}} \log \curlies*{ \frac{\dec{\vec{x}|\vec{z}}}{\enc{\vec{z}|\vec{x}} } } \D{\vec{z}}
\end{equation}

Since the KL-divergence is non-negative the second term (``free energy'') works as a variational lower bound and follows the inequality:

\begin{gather}
	\begin{split}
		\log \dec{\vec{x}} & \ge \mathcal{L}\left(\vec{\theta},\vec{\phi};\vec{x}\right)
		\\ & =
		\int \enc{\vec{z}|\vec{x}} \log \curlies*{ \frac{\dec{\vec{x},\vec{z}}}{\enc{\vec{z}|\vec{x}} } } \D{\vec{z}} \\ 
		& = \E_{\enc{\vec{z}|\vec{x}}} \brackets{- \log \enc{\vec{z}|\vec{x}} + \log \dec{\vec{x},\vec{z}} } 
		\\
		& = -D_{KL}\parens{ \enc{\vec{z}|\vec{x}}||\dec{\vec{z}} } 
		\\ 
		& \quad+ \E_{\enc{\vec{z}|\vec{x}}} \brackets{\log \dec{\vec{x}|\vec{z}} }   	
	\end{split} 
\end{gather}

Maximizing $\mathcal{L}$ wrt. the model parameters, $\phi$ and $\theta$, thereby minimizes the $D_{KL}\left( \enc{\vec{z}|\vec{x}}||\dec{\vec{z}|\vec{x}}\right)$ bringing the . When $D_{KL}\rightarrow 0$ the approximated variational distribution $\enc{\vec{z}|\vec{x}}\rightarrow \dec{\vec{z}|\vec{x}}$

Differentiation of L
Reparameretization trick
Output parameters (mu and sigma)

% TODO: Add algorithm from Kingma13: Algorithm 1 Minibatch version of the Auto-Encoding VB (AEVB) algorithm.


